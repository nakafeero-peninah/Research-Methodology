\documentclass[12pt]{article}
\usepackage{zed-csp}
\usepackage[top=2.5cm, bottom=2.5cm, left=3cm, right=3cm]{geometry}
\usepackage{graphicx}
\begin{document}

\begin{Huge}
\begin{center}
\begin{normalsize}
\textbf{MAKERERE \includegraphics[scale=0.5]{logo} UNIVERSITY }\\


\textbf{FACULTY OF COMPUTING AND INFORMATICS TECHNOLOGY} \\
\textbf{SCHOOL OF COMPUTING AND INFORMATICS TECHNOLOGY} \\
\textbf{DEPARTMENT OF COMPUTER SCIENCE} \\
\textbf{BACHELOR OF SCIENCE IN COMPUTER SCIENCE} \\
\textbf{YEAR 2} \\
\textbf{BIT 2207 RESEARCH METHODOLOGY} \\
\textbf{Course Work: Assignment 1}\\
\end{normalsize}
\end{center}
\end{Huge}

\begin{center}
\begin{tabular}{|l|l|l|c|}
\hline NAME  & REG NO & STD NO \\\hline
Nakafeero Peninah& 16/U/8463/PS & 216009254 \\\hline
\end{tabular}
\paragraph{•}
Lecturer: ERNEST MWEBAZE \\
\end{center}

\newpage

\begin{center}
\textbf{RAMPANT CAUSE OF SEXUAL ASSAULT OF SOCIAL MEDIA PLATFORMS
INTRODUCTION}\\
\end{center}

\section{INTRODUCTION}
\paragraph{•}
It begins with constant texting or stalking on face book, twitter “where are you?”
Students snap funny, humiliating pictures and setup expose pages, websites which earn them more followers especially on instagram. People get excited about seeing students exposed knowing their day to day lives, businesses hence call it a trend.\\
Sexual assault on social media has unfortunately become common places because of cellphones with cameras. Apparently face book, twitter and watsapp are so populated that perverts pedophiles and rapists using the platforms are growing in numbers.\\
This has increased the number of sexual assaults cases on social media sites.

  
\section{OBJECTIVES}
\paragraph{•}
To find out why teenagers misuse social media plat forms.\\
To find out which platforms are highly used as sexual sites.\\
To know the different cases of sexual abuse on social media platforms.\\
To come up with different ways of reducing social media abuse.\\
To track people who misuse social media platforms. 

\section{LITERATURE REVIEW}
\paragraph{•}
Both the usefulness of new media in addressing issues of sexual health and their potential role in placing youth at risk depend critically on the extent with which such media are in use. In 2010, the Kaiser Family Foundation surveyed more than 2,000 young people aged 8-18 years from across the United States regarding their media use. Results indicate that media continue to play a central role in young people's lives. Youth spend a total of 10 hours and 45 minutes each day using various media.

\section{CAUSES}
\paragraph{•}
Face book, twitter, watsapp get hoards of new sign-ups on a daily basis, almost every one on face book having face book accounts is becoming equivalent with accessing internet and that naturally means plenty of criminals are also going to sign up and login.
There is really no fire way for social media e.g. FB, twitter, last control who joins and hence sex offenders are banking on.
\paragraph{•}
Social media having an instant accessibility has made it difficult to deal with content which is hateful, abusive or illegal. This is also due to lack of traditional safeguards of editors and moderators to be able to screen content before it publically viewed.
\paragraph{•}
Openness has become a norm where thousands of selfies, bed snaps are shared on social media. As  a consequence  it becomes  much easier for predators to trick young people into sharing explicit images of latex  them selves as falling prey to an overly sexualized culture.
\paragraph{•}
For example fb has removed its community standards in its takedown policy; it now includes a separate section on dangerous organization and gives more details about what types of nudity it allows to be posted.

\section{CONCLUSION}
\paragraph{•}
Clearly, the story of social media and sexual assault is more complicated than social media simply being used as a platform to fight sexual assault or as an extension of the crime itself. Because, like any other tool, the person using it must decide whether social media is used to hurt or to heal. Educating students about the enormous impact their decisions have on the survivors of sexual assault suggests the importance of harm-prevention training. The more students know about social media’s effect on sexual assault victims — the damage it can cause or, alternatively, its power to solve the problem of sexual assault — the better equipped they’ll be to use social media in way that heals instead of hurts.\\
Always;\\
Remember nothing is private online.\\
Nothing can be taken back.\\
Nothing is truly anonymous.\\
Consider reaction and feelings of the other.



\end{document}